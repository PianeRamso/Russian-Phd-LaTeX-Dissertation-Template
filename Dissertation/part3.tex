\chapter{Что ваще сделано то} \label{chapt3}

\section{Интеграция потоковых моделей метаболизма}

\subsection{Алгоритм загрузки и интеграции модели в базу}

Интеграция модели в базу данных происходит путем разбора файла модели в формате SBML. Для этого реализован программный модуль, который, с использованием библиотеки JSBML, извлекает из файла модели всю информацию, необходимую для описания метаболической сети: данные о ферментах, катализирующих биохимические реакции; участвующие в реакциях низкомолекулярные вещества; стехиометрические коэффициенты; ограничения, накладываемые на значения потоков реакций и компартменты, в которых происходит та или иная реакция. На этапе загрузки модели происходит интеграция данных модели в базу данных, то есть осуществляется определение и создание функциональных связей между существующими в базе и вновь создаваемыми объектами, которые были получены из математической модели. Для идентификации или аннотации конкретных элементов модели (белков, низкомолекулярных веществ, реакций) происходит сопоставление с таковыми объектами в базе данных по нескольким критериям: по ссылкам на внешние базы данных, по полному совпадению названия, включая синонимы \te{сделать блок-схему алгоритма}. В том случае, если поиск в базе увенчался успехом, новый объект не создается и дальнейшая работа ведется с найденным. 
Схема данных, разработанная для хранения элементов математической модели приведена на рисунке. \te{Описать подробно каждый шаг интеграции с картинками, указать какие данные в какое место схемы сохраняются}.
Также определяются реакции, являющиеся спонтанными (т.е. протекают без участия фермента), такие реакции помечаются меткой Spontaneous. Этот этап является важным в контексте автоматической сборки моделей, т. к. основной критерий наличия биохимической реакции в собираемой модели — это присутствие в организме соответствующего фермента, а спонтанные реакции протекают без участия фермента, поэтому такие реакции требуют особой обработки при сборке моделей, а значит, и особой аннотации при интеграции моделей в базу данных.
Другими словами, процесс интеграции модели в базу данных представляет собой преобразование модели метаболической сети в графовое представление, имеющее строго определенную структуру, последующую аннотацию отдельных её элементов для создания связей с существующими в базе объектами, и непосредственно сохранение в базу данных.

\subsection{Алгоритм сборки шаблона модели}

Автоматическая сборка математической модели метаболической сети для одного организма происходит на основании данных о наличии в этом организме ферментов, обладающих способностью катализировать биохимические реакции. Эти реакции определяется на основании гомологии белковых последовательностей в сравнении с другими организмами, для которых в базе данных присутствуют математические модели. Организм, для которого будет собираться модель, далее будем называть целевым.
Алгоритм сборки моделей может быть разбит на несколько этапов:
  \begin{enumerate}
    \item Определяется референсная модель, которая послужит основой для собираемой модели. В рассмотрение принимаются только организмы, для которых в базе имеются математические модели. Весь процесс выполняется автоматически на основании количества гомологичных белков, катализирующих реакции в других организмах: для каждой модели происходит подсчет гомологичных белков, обладающих каталитической активностью и в качестве рефернсной выбирается модель с наибольшим их  количеством.
    \item Из референсной модели удаляются реакции, катализируемые белками, отсутствующими в целевом организме. 
    \item На основании данных о гомологии белков, выбираются реакции присутствующие во всех остальных организмах и добавляются в модель.
    \item Добавление реакции производства биомассы из референсной модели.
    \item Добавление всех спонтанных реакций.
  \end{enumerate}

Используя графовую структуру данных извлекается информация, необходимая для полного описания метаболической сети: стехиометрические коэффициенты реактантов и продуктов реакций, ограничения на потоки реакций, низкомолекулярные вещества, участвующие в реакциях.
Данные, извлеченные из базы данных, интегрируются в единую модель в формате SBML с помощью библиотеки JSBML \te{ссылка}. 
Отдельно стоит отметить добавление реакций, осуществляемых ферментами, которые представляют собой комплекс из нескольких субъединиц. Например, в модели iML1515 из 2712 реакций \te{посчитать} катализируются мультисубъединичными ферментами, поэтому для определения таких реакций необходим отдельный подход. Такие реакции включаются в собираемую модель только при условии, что гомологи всех субъединиц присутствуют в целевом организме. \te{1. Берем все белки, гомологичные белкам целевого организма. 2. Берем реакции с мультисубъединичными ферментами. 3. Отсеиваем те, в которых хотя бы одна субъединица отсутствует в (1).}
В результате на выходе алгоритма получаем модель в формате, пригодном для симуляции в популярных программных пакетах и дальнейшей доработки или уточнения.
\te{обязательна блок-схема}

\subsection{Сборка модели для E. coli str. K-12 substr. W3110}

Для доказательства работоспособности подхода была произведена сборка модели для организма E. coli str. K-12 substr. W3110. Этот подштамм является наиболее близким по филогенетическому дереву к наиболее изученному E. coli str. K-12 substr. MG1655. Модель собиралась на основании данных для двух организмов E. coli str. K-12 substr. MG1655 и для E. coli W, для которых существуют качественные потоковые модели масштаба генома iML1515 \te{ссылка} и iECW\_1372 \te{ссылка}, соответственно. В качестве референсной модели автоматически была выбрана iML1515, так как этот организм более близок по филогении. На собранной модели были проведены симуляции методом FBA, с оптимизацией потока через реакцию биомассы. Модель показала адекватное (близкое к значению, заданного для оптимизации) значение потока через реакцию биомассы после оптимизации. Это означает, что модель является полностью пригодной к симуляциям в любых предназначенных для этого программных пакетах и является консистентной с биологической точки зрения.
Далее было проведено сравнение с ранее созданной и доступной в базе BiGG моделью целевого организма -- iY75\_1357 \te{ссылка}. Поведение моделей в наиболее консервативных метаболических путях отличалось незначительно, но, тем не менее, в некоторых участках расхождения оказались существенными. Так, после проведенного анализа было выяснено, что в геноме E. coli K-12 substr. W3110 в NCBI было удалено описание гена fdhF, что сделало нефункциональным фермент formate dehydrogenase-H и привело к заметным отклонениям в поведении построенной модели от модели из базы данных BiGG. Также для обеих моделей был проведен анализ методом FVA. Реакции, наиболее сильно различающиеся по диапазонам значений потоков, представлены в таблице \te{сделать таблицу}. Несмотря на некоторые различия в поведении моделей, предложенный подход для генерации шаблонов потоковых моделей метаболизма масштаба генома можно считать полностью работоспособным.
Стоит отметить, что сборка производилась на основании данных и моделей для всего лишь двух организмов, и даже при столь небольшом количестве исходных данных собранная модель оказалась консистентной и пригодной для симуляции. На основании этого можно утверждать, что при использовании исходных данных для большего количество организмов, качество собираемых моделей будет увеличиваться, тем самым еще больше сокращая время разработки финальных моделей.

\section{Таблица обыкновенная} \label{sect3_1}

Так размещается таблица:

\begin{table} [htbp]
  \centering
  \changecaptionwidth\captionwidth{15cm}
  \caption{Название таблицы}\label{Ts0Sib}%
  \begin{tabular}{| p{3cm} || p{3cm} | p{3cm} | p{4cm}l |}
  \hline
  \hline
  Месяц   & \centering $T_{min}$, К & \centering $T_{max}$, К &\centering  $(T_{max} - T_{min})$, К & \\
  \hline
  Декабрь &\centering  253.575   &\centering  257.778    &\centering      4.203  &   \\
  Январь  &\centering  262.431   &\centering  263.214    &\centering      0.783  &   \\
  Февраль &\centering  261.184   &\centering  260.381    &\centering     $-$0.803  &   \\
  \hline
  \hline
  \end{tabular}
\end{table}

\begin{table} [htbp]% Пример записи таблицы с номером, но без отображаемого наименования
    \centering
    \parbox{9cm}{% чтобы лучше смотрелось, подбирается самостоятельно
        \captiondelim{}% должен стоять до самого пустого caption
        \caption{}%
        \label{tbl:test1}%
        \begin{SingleSpace}
            \begin{tabular}{| c | c | c | c |}
                \hline
                Оконная функция & ${2N}$& ${4N}$& ${8N}$\\ \hline
                Прямоугольное   & 8.72  & 8.77  & 8.77  \\ \hline
                Ханна           & 7.96  & 7.93  & 7.93  \\ \hline
                Хэмминга        & 8.72  & 8.77  & 8.77  \\ \hline
                Блэкмана        & 8.72  & 8.77  & 8.77  \\ \hline
            \end{tabular}%
        \end{SingleSpace}
    }
\end{table}

Таблица \ref{tbl:test2} "--- пример таблицы, оформленной в~классическом книжном
варианте или~очень близко к~нему. \mbox{ГОСТу} по~сути не~противоречит. Можно
ещё~улучшить представление, с~помощью пакета \verb|siunitx| или~подобного.

\begin{table} [htbp]%
    \centering
    \caption{Наименование таблицы, очень длинное наименование таблицы, чтобы посмотреть как оно будет располагаться на~нескольких строках и~переноситься}%
    \label{tbl:test2}% label всегда желательно идти после caption
    \renewcommand{\arraystretch}{1.5}%% Увеличение расстояния между рядами, для улучшения восприятия.
    \begin{SingleSpace}
        \begin{tabular}{@{}@{\extracolsep{20pt}}llll@{}} %Вертикальные полосы не используются принципиально, как и лишние горизонтальные (допускается по ГОСТ 2.105 пункт 4.4.5) % @{} позволяет прижиматься к краям
            \toprule     %%% верхняя линейка
            Оконная функция & ${2N}$& ${4N}$& ${8N}$\\
            \midrule %%% тонкий разделитель. Отделяет названия столбцов. Обязателен по ГОСТ 2.105 пункт 4.4.5 
            Прямоугольное   & 8.72  & 8.77  & 8.77  \\
            Ханна           & 7.96  & 7.93  & 7.93  \\
            Хэмминга        & 8.72  & 8.77  & 8.77  \\
            Блэкмана        & 8.72  & 8.77  & 8.77  \\
            \bottomrule %%% нижняя линейка
        \end{tabular}%
    \end{SingleSpace}
\end{table}

\section{Таблица с многострочными ячейками и примечанием}

Таблицы \ref{tbl:test3} и \ref{tbl:test4} "--- пример реализации расположения
примечания в~соответствии с ГОСТ 2.105. Каждый вариант со своими достоинствами
и~недостатками. Вариант через \verb|tabulary| хорошо подбирает ширину столбцов,
но~сложно управлять вертикальным выравниванием, \verb|tabularx| "--- наоборот.
\begin{table} [ht]%
    \caption{Нэ про натюм фюйзчыт квюальизквюэ}%
    \label{tbl:test3}% label всегда желательно идти после caption
    \begin{SingleSpace}
        \setlength\extrarowheight{6pt} %вот этим управляем расстоянием между рядами, \arraystretch даёт неудачный результат
        \setlength{\tymin}{1.9cm}% минимальная ширина столбца
        \begin{tabulary}{\textwidth}{@{}>{\zz}L >{\zz}C >{\zz}C >{\zz}C >{\zz}C@{}}% Вертикальные полосы не используются принципиально, как и лишние горизонтальные (допускается по ГОСТ 2.105 пункт 4.4.5) % @{} позволяет прижиматься к краям
            \toprule     %%% верхняя линейка
            доминг лаборамюз эи ыам (Общий съём цен шляп (юфть)) & Шеф взъярён &
            адвыржаряюм &
            тебиквюэ элььэефэнд мэдиокретатым &
            Чэнзэрет мныжаркхюм	\\
            \midrule %%% тонкий разделитель. Отделяет названия столбцов. Обязателен по ГОСТ 2.105 пункт 4.4.5 
            Эй, жлоб! Где туз? Прячь юных съёмщиц в~шкаф Плюш изъят. Бьём чуждый цен хвощ! &
            ${\approx}$ &
            ${\approx}$ &
            ${\approx}$ &
            $ + $ \\
            Эх, чужак! Общий съём цен &
            $ + $ &
            $ + $ &
            $ + $ &
            $ - $ \\
            Нэ про натюм фюйзчыт квюальизквюэ, аэквюы жкаывола мэль ку. Ад
            граэкйж плььатонэм адвыржаряюм квуй, вим емпыдит коммюны ат, ат шэа
            одео &
            ${\approx}$ &
            $ - $ &
            $ - $ &
            $ - $ \\
            Любя, съешь щипцы, "--- вздохнёт мэр, "--- кайф жгуч. &
            $ - $ &
            $ + $ &
            $ + $ &
            ${\approx}$ \\
            Нэ про натюм фюйзчыт квюальизквюэ, аэквюы жкаывола мэль ку. Ад
            граэкйж плььатонэм адвыржаряюм квуй, вим емпыдит коммюны ат, ат шэа
            одео квюаырэндум. Вёртюты ажжынтиор эффикеэнди эож нэ. &
            $ + $ &
            $ - $ &
            ${\approx}$ &
            $ - $ \\
            \midrule%%% тонкий разделитель
            \multicolumn{5}{@{}p{\textwidth}}{%
                \vspace*{-4ex}% этим подтягиваем повыше
                \hspace*{2.5em}% абзацный отступ - требование ГОСТ 2.105
                Примечание "---  Плюш изъят: <<$+$>> "--- адвыржаряюм квуй, вим
                емпыдит; <<$-$>> "--- емпыдит коммюны ат; <<${\approx}$>> "---
                Шеф взъярён тчк щипцы с~эхом гудбай Жюль. Эй, жлоб! Где туз?
                Прячь юных съёмщиц в~шкаф. Экс-граф?
            }
            \\
            \bottomrule %%% нижняя линейка
        \end{tabulary}%
    \end{SingleSpace}
\end{table}

Если таблица \ref{tbl:test3} не помещается на той же странице, всё
её~содержимое переносится на~следующую, ближайшую, а~этот текст идёт перед ней.
\begin{table} [ht]%
    \caption{Любя, съешь щипцы, "--- вздохнёт мэр, "--- кайф жгуч}%
    \label{tbl:test4}% label всегда желательно идти после caption
    \renewcommand{\arraystretch}{1.6}%% Увеличение расстояния между рядами, для улучшения восприятия.
    \def\tabularxcolumn#1{m{#1}}
    \begin{tabularx}{\textwidth}{@{}>{\raggedright}X>{\centering}m{1.9cm} >{\centering}m{1.9cm} >{\centering}m{1.9cm} >{\centering\arraybackslash}m{1.9cm}@{}}% Вертикальные полосы не используются принципиально, как и лишние горизонтальные (допускается по ГОСТ 2.105 пункт 4.4.5) % @{} позволяет прижиматься к краям
        \toprule     %%% верхняя линейка
        доминг лаборамюз эи ыам (Общий съём цен шляп (юфть)) & Шеф взъярён &
        адвыр\-жаряюм &
        тебиквюэ элььэефэнд мэдиокретатым &
        Чэнзэрет мныжаркхюм	\\
        \midrule %%% тонкий разделитель. Отделяет названия столбцов. Обязателен по ГОСТ 2.105 пункт 4.4.5 
        Эй, жлоб! Где туз? Прячь юных съёмщиц в~шкаф Плюш изъят.
        Бьём чуждый цен хвощ! &
        ${\approx}$ &
        ${\approx}$ &
        ${\approx}$ &
        $ + $ \\
        Эх, чужак! Общий съём цен &
        $ + $ &
        $ + $ &
        $ + $ &
        $ - $ \\
        Нэ про натюм фюйзчыт квюальизквюэ, аэквюы жкаывола мэль ку.
        Ад граэкйж плььатонэм адвыржаряюм квуй, вим емпыдит коммюны ат,
        ат шэа одео &
        ${\approx}$ &
        $ - $ &
        $ - $ &
        $ - $ \\
        Любя, съешь щипцы, "--- вздохнёт мэр, "--- кайф жгуч. &
        $ - $ &
        $ + $ &
        $ + $ &
        ${\approx}$ \\
        Нэ про натюм фюйзчыт квюальизквюэ, аэквюы жкаывола мэль ку. Ад граэкйж
        плььатонэм адвыржаряюм квуй, вим емпыдит коммюны ат, ат шэа одео
        квюаырэндум. Вёртюты ажжынтиор эффикеэнди эож нэ. &
        $ + $ &
        $ - $ &
        ${\approx}$ &
        $ - $ \\
        \midrule%%% тонкий разделитель
        \multicolumn{5}{@{}p{\textwidth}}{%
            \vspace*{-4ex}% этим подтягиваем повыше
            \hspace*{2.5em}% абзацный отступ - требование ГОСТ 2.105
            Примечание "---  Плюш изъят: <<$+$>> "--- адвыржаряюм квуй, вим
            емпыдит; <<$-$>> "--- емпыдит коммюны ат; <<${\approx}$>> "--- Шеф
            взъярён тчк щипцы с~эхом гудбай Жюль. Эй, жлоб! Где туз? Прячь юных
            съёмщиц в~шкаф. Экс-граф?
        }
        \\
        \bottomrule %%% нижняя линейка
    \end{tabularx}%
\end{table}

\section{Параграф "--- два} \label{sect3_2}

Некоторый текст.

\section{Параграф с подпараграфами} \label{sect3_3}

\subsection{Подпараграф "--- один} \label{subsect3_3_1}

Некоторый текст.

\subsection{Подпараграф "--- два} \label{subsect3_3_2}

Некоторый текст.

\clearpage